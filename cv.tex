%%%%%%%%%%%%%%%%%%%%%%%%%%%%%%%%%%%%%%%%%
% "ModernCV" CV and Cover Letter
% LaTeX Template
% Version 1.1 (9/12/12)
%
% This template has been downloaded from:
% http://www.LaTeXTemplates.com
%
% Original author:
% Xavier Danaux (xdanaux@gmail.com)
%
% License:
% CC BY-NC-SA 3.0 (http://creativecommons.org/licenses/by-nc-sa/3.0/)
%
% Important note:
% This template requires the moderncv.cls and .sty files to be in the same 
% directory as this .tex file. These files provide the resume style and themes 
% used for structuring the document.
%
%%%%%%%%%%%%%%%%%%%%%%%%%%%%%%%%%%%%%%%%%

%----------------------------------------------------------------------------------------
%   PACKAGES AND OTHER DOCUMENT CONFIGURATIONS
%----------------------------------------------------------------------------------------

\documentclass[11pt,a4paper,roman]{moderncv} % Font sizes: 10, 11, or 12; paper sizes: a4paper, letterpaper, a5paper, legalpaper, executivepaper or landscape; font families: sans or roman


\moderncvstyle{classic} % CV theme - options include: 'casual' (default), 'classic', 'oldstyle' and 'banking'
\moderncvcolor{black} % CV color - options include: 'blue' (default), 'orange', 'green', 'red', 'purple', 'grey' and 'black'

\usepackage{lipsum} % Used for inserting dummy 'Lorem ipsum' text into the template

\usepackage[scale=0.75]{geometry} % Reduce document margins
%\setlength{\hintscolumnwidth}{3cm} % Uncomment to change the width of the dates column
%\setlength{\makecvtitlenamewidth}{10cm} % For the 'classic' style, uncomment to adjust the width of the space allocated to your name


%\usepackage[english]{babel}
%\usepackage{hyperref}
\usepackage[utf8]{inputenc}
\usepackage[T1]{fontenc}
\usepackage{color}
\usepackage[english]{babel}

%----------------------------------------------------------------------------------------
%   NAME AND CONTACT INFORMATION SECTION
%----------------------------------------------------------------------------------------

\firstname{André} % Your first name
\familyname{F. Rendeiro} % Your last name

% All information in this block is optional, comment out any lines you don't need
\title{Curriculum Vitae}
%\address{Lazarettgasse, 14/B1114 1090 Vienna, Austria}
%\mobile{+43 6502011854}
%\phone{(000) 111 1112}
%\fax{(000) 111 1113}
\email{arendeiro@cemm.oeaw.ac.at}
\homepage{andre-rendeiro.me} % The first argument is the url for the clickable link, the second argument is the url displayed in the template - this allows special characters to be displayed such as the tilde in this example
%\extrainfo{additional information}
%\photo[70pt][0.4pt]{/home/afr/Pictures/me.png} % The first bracket is the picture height, the second is the thickness of the frame around the picture (0pt for no frame)
%\quote{"A witty and playful quotation" - John Smith}

%----------------------------------------------------------------------------------------

\begin{document}

\makecvtitle % Print the CV title

%----------------------------------------------------------------------------------------
%   CURRENT POSITION
%----------------------------------------------------------------------------------------

\section{Current position}
\cventry{2014-present}
	{PhD student}{CeMM Research Centre for Molecular Medicine of the Austrian Academy of Sciences}{Vienna, Austria}
	{Christoph Bock's lab}{}

%----------------------------------------------------------------------------------------
%   EDUCATION SECTION
%----------------------------------------------------------------------------------------

\section{Education}

	\cventry{2012-2014}{Masters in Molecular and Cell Biology}{University of Aveiro}{Portugal}{}{}

	\subsection{Thesis}
		\cvitem{Title}{\textit{Regulation of Oikopleura dioica's alternative cell cycle modes}}
		\cvitem{Supervisor}{Professor Eric Thompson}

	\cventry{2008-2012}{Bachelor in Biology}{University of Aveiro}{Portugal}{}{}
%----------------------------------------------------------------------------------------
%   WORK EXPERIENCE SECTION
%----------------------------------------------------------------------------------------

\section{Experience}

	\subsection{Scientific Activity}

	\cventry{2014-present}
		{PhD student}
		{CeMM Research Centre for Molecular Medicine of the Austrian Academy of Sciences, Vienna, Austria}{Christoph Bock's lab}
		{}
		{}
	
	\cventry{2013-2014}
		{The role of E2F regulation and H3K79 methylation in \textit{Oikopleura dioica}'s cell cycle modes}
		{Sars International Centre for Marine Molecular Biology, Bergen, Norway}{Eric Thompson's lab}
		{}
		{I investigated the molecular mechanisms of alternative cell cycle modes (particularly endocycles) in the chordate \textit{Oikopleura dioica} by performing ChIP-seq on transcription factors involved in cell cycle regulation (E2F). I also studied the role of histone 3 lysine 79 methylation on cell cycle regulation through functional studies on its methyltransferase, Dot1.}
	
	\cventry{2011-2012}
		{Identification of cis-regulatory elements in \textit{Nematostella vectensis} using ChIP-seq}
		{Dept. of Molecular Evolution and Development, University of Vienna, Austria}{Uli Technau's lab}
		{}
		{I performed ChIP-seq of chromatin modifications and other regulatory proteins over several developmental stages of \textit{Nematostella vectensis}, constructed a map of chromatin states and predicted cis-regulatory elements genome-wide. I also tested the function of some of these regions \textit{in vivo} in a reporter assay.}

	\cventry{2010-2011}
		{Tol2-mediated zebrafish transgenesis for studies in protein mistranslation}
		{RNA Biology Laboratory, Biology Department, University of Aveiro, Portugal}{Manuel Santos' lab}
		{}
		{I created transgenic zebrafish that were used as a model for studies in neurodegeneration through protein aggregation. This was caused by increasing the level of translational error (mistranslation) during endogenous protein synthesis. I learned to build plasmid constructs, microinject them in zebrafish and screen for phenotypes.}

	\cventry{2009-2010}
		{Transciptome studies with microarrays in heat-shocked yeast}
		{RNA Biology Laboratory, Biology Department, University of Aveiro, Portugal}{Manuel Santos' lab}
		{}
		{I was involved in the analysis of microarray expression data from yeast under various treatments. I learned to pre-process, normalize and explore data programmatically to detect significant differential gene expression, clustering genes and exploring their gene ontology across treatments.}

%----------------------------------------------------------------------------------------
%   PUBLICATIONS
%----------------------------------------------------------------------------------------

\section{Publications}
	\cvitem{Peer reviewed}{
		\begin{enumerate}
			\item Christian Schmidl*,\underline{André F. Rendeiro}*,  Nathan C Sheffield, Christoph Bock. 2015. \textbf{ChIPmentation: fast, robust, low-input ChIP-seq for histones and transcription factors}. Nature Methods. doi:10.1038/nmeth.3542
			\item Michaela Schwaiger, Anna Schönauer, \underline{André F. Rendeiro}, Carina Pribitzer, Alexandra Schauer, Anna Gilles, Johannes Schinko, David Fredman, and Ulrich Technau. \textbf{Evolutionary conservation of the eumetazoan gene regulatory landscape}. Genome Research, 1–12. doi:10.1101/gr.162529.113
		\end{enumerate}
	}
	\cvitem{}{
		* \textit{equal contributions}
	}
	
	\cvitem{Non-peer reviewed}{
		\begin{enumerate}
			\item \underline{André F. Rendeiro}, Pavla  Navratilova, Eric Thompson (2014). \textbf{Chromatin preparation for ChIP-seq in \textit{Oikopleura dioica}}. figshare. http://dx.doi.org/10.6084/m9.figshare.884562
		\end{enumerate}
	}

\section{Communications}
	
	\cvitem{Conference talks}{
		\begin{enumerate}
			\item Michaela Schwaiger, Anna Schönauer, \underline{André F. Rendeiro}, Carina Pribitzer, Alexandra Schauer, Anna Gilles, Johannes Schinko, David Fredman, and Ulrich Technau. \textbf{Evolutionary conservation of the eumetazoan gene regulatory landscape}. \textit{XVIII Portuguese Genetics Society Meeting}, June 2013. Porto, Portugal
		\end{enumerate}
	}
	\cvitem{Conference posters}{
		\begin{enumerate}
			\item Anna Schönauer, \underline{André F. Rendeiro}, Michaela Schwaiger, Ulrich Technau. \textbf{Identification of cis-regulatory elements in the sea anemone \textit{Nematostella vectensis}}. \textit{Evonet Symposium}, September 2012, Vienna Austria. http://dx.doi.org/10.6084/m9.figshare.107026
		\end{enumerate}
	}
	
%----------------------------------------------------------------------------------------
%   SKILLS SECTION
%----------------------------------------------------------------------------------------

\section{Skills}

\subsection{Computational}
\cvitemwithcomment{Programming languages}{\textnormal{Python, R, Perl, C/C++}}{In this order of proficiency}
\cvitem{Web development}{HTML, CSS, PHP, Javascript, Django, Wordpress}
\cvitem{Bioinformatics}{ChIP-seq/ATAC-seq/DNase-seq data analysis; gene expression data analysis (microarray and RNA-seq); \textit{de novo} transcriptome assembly and annotation; method implementation}

\subsection{Molecular Biology}
\cvitem{Techniques}{Zebrafish chemical screening, Chromatin imunoprecipitation (ChIP), Library preparation, Western and Northern blotting, PCR, qRT-PCR, SDM PCR, molecular cloning, zebrafish and \textit{Nematostella} microinjection, immunohistochemistry/fluorecence and confocal microscopy}

%----------------------------------------------------------------------------------------
%   EXTRA COURSES
%----------------------------------------------------------------------------------------
\section{Advanced courses}
\cvitem{2011}{Scientific writing course (Maria Dornelas - University of St. Andrews)}

%----------------------------------------------------------------------------------------
%   AWARDS SECTION
%----------------------------------------------------------------------------------------

\section{Awards/Scholarships}

	\cventry{2013-2014}
		{Erasmus studies mobility program scholarship}{}{}{}{European Commission}
	\cventry{2011-2012}
		{Erasmus intership mobility program scholarship}{}{}{}{European Commission}
	\cventry{2009-2010}
		{'Integration into Research' Grant}{}{}{}{Science and Technology Foundation - Portugal}

%----------------------------------------------------------------------------------------
%   MEMBERSHIPS
%----------------------------------------------------------------------------------------

\section{Associative/Administrative positions}
\cvitem{2010-2012}{Member of the Biology department counsel, University of Aveiro}
\cvitem{2009-2011}{Member of the undergraduate Biology committee, University of Aveiro}

%----------------------------------------------------------------------------------------
%   LANGUAGES SECTION
%----------------------------------------------------------------------------------------

\section{Languages}

\cvitemwithcomment{Portuguese}{\textnormal{Native speaker}}{}
\cvitemwithcomment{English}{\textnormal{Very good}}{Fluent}
\cvitemwithcomment{Spanish}{\textnormal{Conversational}}{}
\cvitemwithcomment{German}{\textnormal{Basic}}{Basic words and phrases only}
\cvitemwithcomment{French}{\textnormal{Basic}}{Basic words and phrases only}

%----------------------------------------------------------------------------------------
%   INTERESTS SECTION
%----------------------------------------------------------------------------------------

\section{Other interests}

\cvlistdoubleitem{Classical singing}{Opera}
\cvlistdoubleitem{Choir conducting}{Piano}
\cvlistdoubleitem{Literature}{Cinema}
\cvlistitem{Coding websites and web apps}

%----------------------------------------------------------------------------------------
%   COVER LETTER
%----------------------------------------------------------------------------------------

% To remove the cover letter, comment out this entire block

%\clearpage

%\recipient{HR Departmnet}{Corporation\\123 Pleasant Lane\\12345 City, State} % Letter recipient
%\date{\today} % Letter date
%\opening{Dear Sir or Madam,} % Opening greeting
%\closing{Sincerely yours,} % Closing phrase
%\enclosure[Attached]{curriculum vit\ae{}} % List of enclosed documents

%\makelettertitle % Print letter title

%\lipsum[1-3] % Dummy text

%\makeletterclosing % Print letter signature

%----------------------------------------------------------------------------------------

\end{document}