%%%%%%%%%%%%%%%%%%%%%%%%%%%%%%%%%%%%%%%%%
% "ModernCV" CV and Cover Letter
% LaTeX Template
% Version 1.1 (9/12/12)
%
% This template has been downloaded from:
% http://www.LaTeXTemplates.com
%
% Original author:
% Xavier Danaux (xdanaux@gmail.com)
%
% License:
% CC BY-NC-SA 3.0 (http://creativecommons.org/licenses/by-nc-sa/3.0/)
%
% Important note:
% This template requires the moderncv.cls and .sty files to be in the same 
% directory as this .tex file. These files provide the resume style and themes 
% used for structuring the document.
%
%%%%%%%%%%%%%%%%%%%%%%%%%%%%%%%%%%%%%%%%%

%----------------------------------------------------------------------------------------
%   PACKAGES AND OTHER DOCUMENT CONFIGURATIONS
%----------------------------------------------------------------------------------------

\documentclass[11pt,a4paper,roman]{moderncv} % Font sizes: 10, 11, or 12; paper sizes: a4paper, letterpaper, a5paper, legalpaper, executivepaper or landscape; font families: sans or roman


\moderncvstyle{classic} % CV theme - options include: 'casual' (default), 'classic', 'oldstyle' and 'banking'
\moderncvcolor{black} % CV color - options include: 'blue' (default), 'orange', 'green', 'red', 'purple', 'grey' and 'black'

\usepackage{lipsum} % Used for inserting dummy 'Lorem ipsum' text into the template

\usepackage[scale=0.75]{geometry} % Reduce document margins
%\setlength{\hintscolumnwidth}{3cm} % Uncomment to change the width of the dates column
%\setlength{\makecvtitlenamewidth}{10cm} % For the 'classic' style, uncomment to adjust the width of the space allocated to your name


%\usepackage[english]{babel}
%\usepackage{hyperref}
\usepackage[utf8]{inputenc}
\usepackage[T1]{fontenc}
\usepackage{color}
\usepackage[english]{babel}


% footer page count
\usepackage{lastpage}
\usepackage{fancyhdr}
\pagestyle{fancy} 

\cfoot{\thepage\ of \pageref{LastPage}}

%----------------------------------------------------------------------------------------
%   NAME AND CONTACT INFORMATION SECTION
%----------------------------------------------------------------------------------------

\firstname{André} % Your first name
\familyname{F. Rendeiro} % Your last name

% All information in this block is optional, comment out any lines you don't need
\title{Curriculum Vitae}
%\address{Lazarettgasse, 14/B1114 1090 Vienna, Austria}
%\mobile{+43 6502011854}
%\phone{(000) 111 1112}
%\fax{(000) 111 1113}
\email{arendeiro@cemm.oeaw.ac.at}
\homepage{andre-rendeiro.com} % The first argument is the url for the clickable link, the second argument is the url displayed in the template - this allows special characters to be displayed such as the tilde in this example
%\extrainfo{additional information}
%\photo[70pt][0.4pt]{/home/afr/Pictures/me.png} % The first bracket is the picture height, the second is the thickness of the frame around the picture (0pt for no frame)
%\quote{"A witty and playful quotation" - John Smith}

%----------------------------------------------------------------------------------------

\begin{document}

\makecvtitle % Print the CV title

%----------------------------------------------------------------------------------------
%   CURRENT POSITION
%----------------------------------------------------------------------------------------

\section{Current position}
\cventry{2014-present}
    {PhD student}{CeMM Research Centre for Molecular Medicine of the Austrian Academy of Sciences}{Vienna, Austria}
    {Christoph Bock's lab}{}

%----------------------------------------------------------------------------------------
%   EDUCATION SECTION
%----------------------------------------------------------------------------------------

\section{Education}
    \cventry{2012-2014}{Masters in Molecular and Cell Biology}{University of Aveiro}{Portugal}{}{}
    \cventry{2008-2012}{Bachelor in Biology}{University of Aveiro}{Portugal}{}{}

%----------------------------------------------------------------------------------------
%   PUBLICATIONS
%----------------------------------------------------------------------------------------

\section{Publications}
    \cvitem{Peer reviewed}{
        \item Paul Datlinger, \underline{André F Rendeiro}*, Christian Schmidl*, Thomas Krausgruber, Peter Traxler, Johanna Klughammer, Linda C Schuster, Amelie Kuchler, Donat Alpar, Christoph Bock. \textbf{Pooled CRISPR screening with single-cell transcriptome readout}. Nature Methods. (2017) doi:10.1038/nmeth.4177
        \item Roman A Romanov, Amit Zeisel, Joanne Bakker, Fatima Girach, Arash Hellysaz, Raju Tomer, Alán Alpár, Jan Mulder, Frédéric Clotman, Erik Keimpema, Brian Hsueh, Ailey K Crow, Henrik Martens, Christian Schwindling, Daniela Calvigioni, Jaideep S Bains, Zoltán Máté, Gábor Szabó, Yuchio Yanagawa, Ming-Dong Zhang, \underline{Andre Rendeiro}, Matthias Farlik, Mathias Uhlén, Peer Wulff, Christoph Bock, Christian Broberger, Karl Deisseroth, Tomas Hökfelt, Sten Linnarsson, Tamas L Horvath, Tibor Harkany. \textbf{Molecular interrogation of hypothalamic organization reveals distinct dopamine neuronal subtypes}. Nature Neuroscience. (2016) doi:10.1038/nn.4462
        \item Clara Jana-Lui Busch, Tim Hendrikx, David Weismann, Sven Jäckel, Sofie M. A. Walenbergh, \underline{André F. Rendeiro}, Juliane Weißer, Florian Puhm, Anastasiya Hladik, Laura Göderle, Nikolina Papac-Milicevic, Gerald Haas, Vincent Millischer, Saravanan Subramaniam, Sylvia Knapp, Keiryn L. Bennett, Christoph Bock, Christoph Reinhardt, Ronit Shiri-Sverdlov, Christoph J. Binder. \textbf{Malondialdehyde epitopes are sterile mediators of hepatic inflammation in hypercholesterolemic mice}. Hepatology. (2016) doi:10.1002/hep.28970
        \item \underline{André F Rendeiro}*, Christian Schmidl*, Jonathan C. Strefford*, Renata Walewska, Zadie Davis, Matthias Farlik, David Oscier, Christoph Bock. \textbf{Chromatin accessibility maps of chronic lymphocytic leukaemia identify subtype-specific epigenome signatures and transcription regulatory networks}. Nature Communications. 7:11938 (2016) doi:10.1038/ncomms11938
    }
    \cvitem{}{
        \item Christian Schmidl*,\underline{André F. Rendeiro}*,  Nathan C Sheffield, Christoph Bock. 2015. \textbf{ChIPmentation: fast, robust, low-input ChIP-seq for histones and transcription factors}. Nature Methods. doi:10.1038/nmeth.3542
        \item Michaela Schwaiger, Anna Schönauer, \underline{André F. Rendeiro}, Carina Pribitzer, Alexandra Schauer, Anna Gilles, Johannes Schinko, David Fredman, and Ulrich Technau. \textbf{Evolutionary conservation of the eumetazoan gene regulatory landscape}. Genome Research, 1–12. doi:10.1101/gr.162529.113
    }
    \cvitem{}{
        * \textit{equal contributions}
    }
    \cvitem{Non-peer reviewed}{
            \item \underline{André F. Rendeiro}, Pavla  Navratilova, Eric Thompson (2014). \textbf{Chromatin preparation for ChIP-seq in \textit{Oikopleura dioica}}. figshare. http://dx.doi.org/10.6084/m9.figshare.884562
    }

\section{Communications}
    \cvitem{Conference talks}{
            %\item Paul Datlinger, \underline{André F Rendeiro}*, Christian Schmidl*, Thomas Krausgruber, Peter Traxler, Johanna Klughammer, Linda C Schuster, Amelie Kuchler, Donat Alpar, Christoph Bock. \textbf{Pooled CRISPR screening with single-cell transcriptome readout}. \textit{Ascona Workshop 2017}, May 2017, Ascona, Switzerland.
            %\item \underline{André F Rendeiro}. \textbf{Large-scale chromatin profiling uncovers heterogeneity of molecular phenotypes and gene regulatory networks of chronic lymphocytic leukemia}. \textit{Illumina User Meeting}, February 2017, Cologne, Germany.
            \item Michaela Schwaiger, Anna Schönauer, \underline{André F. Rendeiro}, Carina Pribitzer, Alexandra Schauer, Anna Gilles, Johannes Schinko, David Fredman, and Ulrich Technau. \textbf{Evolutionary conservation of the eumetazoan gene regulatory landscape}. \textit{XVIII Portuguese Genetics Society Meeting}, June 2013. Porto, Portugal
    }
    \cvitem{Conference posters}{
            \item \underline{André F Rendeiro}*, Christian Schmidl*, Jonathan C. Strefford*, Renata Walewska, Zadie Davis, Matthias Farlik, David Oscier, Christoph Bock. \textbf{Large-scale chromatin profiling uncovers heterogeneity of molecular phenotypes and gene regulatory networks of chronic lymphocytic leukemia}. \textit{Young Scientist Association of the Medical University of Vienna PhD Symposia}, June 2016, Vienna, Austria. https://doi.org/10.6084/m9.figshare.3479528.v1 \textbf{Best poster award in "Malignant Diseases" category}.
            \item \underline{André F Rendeiro}*, Christian Schmidl*, Jonathan C. Strefford*, Renata Walewska, Zadie Davis, Matthias Farlik, David Oscier, Christoph Bock. \textbf{Large-scale chromatin profiling uncovers heterogeneity of molecular phenotypes and gene regulatory networks of chronic lymphocytic leukemia}. \textit{Keystone Symposia on Chromatin and Epigenetics}, March 2016, Whistler, Vancouver, Canada. https://doi.org/10.6084/m9.figshare.3479528.v1
            \item Anna Schönauer, \underline{André F. Rendeiro}, Michaela Schwaiger, Ulrich Technau. \textbf{Identification of cis-regulatory elements in the sea anemone \textit{Nematostella vectensis}}. \textit{Evonet Symposium}, September 2012, Vienna Austria. http://dx.doi.org/10.6084/m9.figshare.107026
    }
    \cvitem{}{
        * \textit{equal contributions}
    }
    
%----------------------------------------------------------------------------------------
%   SKILLS SECTION
%----------------------------------------------------------------------------------------
\newpage

\section{Skills}

\subsection{Computational}
    \cvitem{Programming languages}{Python, R}
    \cvitem{Bioinformatics}{ATAC-seq/ChIP-seq/RNA-seq data analysis; single-cell RNA-seq analysis; Machine learning; Software development}
    \cvitem{Web development}{Flask/Django, Javascript}

\subsection{Molecular Biology}
\cvitem{Techniques}{Chromatin imunoprecipitation (ChIP), library preparation, Western and Northern blotting, PCR, molecular cloning, chemical screening, zebrafish and \textit{Nematostella} handling and microinjection, immunohistochemistry, fluorecence and confocal microscopy}

%----------------------------------------------------------------------------------------
%   WORK EXPERIENCE SECTION
%----------------------------------------------------------------------------------------

\section{Additional experience}

    \subsection{Scientific Activity}

        \cventry{2013-2014}
            {The role of E2F regulation and H3K79 methylation in \textit{Oikopleura dioica}'s cell cycle modes}
            {Sars International Centre for Marine Molecular Biology, Bergen, Norway}{Eric Thompson's lab}
            {}
            {I investigated the molecular mechanisms of alternative cell cycle modes (particularly endocycles) in the chordate \textit{Oikopleura dioica} by performing ChIP-seq on transcription factors involved in cell cycle regulation (E2F). I also studied the role of H3K79me on cell cycle regulation through functional studies on its methyltransferase, Dot1.}
        
        \cventry{2011-2012}
            {Identification of cis-regulatory elements in \textit{Nematostella vectensis} using ChIP-seq}
            {Dept. of Molecular Evolution and Development, University of Vienna, Austria}{Uli Technau's lab}
            {}
            {I performed ChIP-seq of chromatin modifications and other regulatory proteins over several developmental stages of \textit{Nematostella vectensis}, constructed a map of chromatin states and predicted cis-regulatory elements genome-wide. I also tested the function of some of these regions \textit{in vivo} in a reporter assay.}

        \cventry{2010-2011}
            {Tol2-mediated zebrafish transgenesis for studies in protein mistranslation}
            {RNA Biology Laboratory, Biology Department, University of Aveiro, Portugal}{Manuel Santos' lab}
            {}
            {I created transgenic zebrafish that were used as a model for studies in neurodegeneration through protein aggregation. This was caused by increasing the level of translational error (mistranslation) during endogenous protein synthesis. I learned to build plasmid constructs, microinject them in zebrafish and screen for phenotypes.}

        \cventry{2009-2010}
            {Transciptome studies with microarrays in heat-shocked yeast}
            {RNA Biology Laboratory, Biology Department, University of Aveiro, Portugal}{Manuel Santos' lab}
            {}
            {I was involved in the analysis of microarray expression data from yeast under various treatments. I learned to pre-process, normalise and explore data programmatically to detect significant differential gene expression, clustering genes and exploring their ontology across treatments.}

    \newpage
    \subsection{Associative/Administrative}

        \cvitem{2010-2012}{Member of the Biology department counsel, University of Aveiro}
        \cvitem{2009-2011}{Member of the undergraduate Biology committee, University of Aveiro}

    %----------------------------------------------------------------------------------------
    %   EXTRA COURSES
    %----------------------------------------------------------------------------------------
    \subsection{Advanced courses}
        \cvitem{September 2015}{Summer School on Machine Learning for Personalised Medicine - Marie Curie Initial Training Network, Manchester, UK}
        \cvitem{September 2012}{Scientific writing course - University of Aveiro}

%----------------------------------------------------------------------------------------
%   AWARDS SECTION
%----------------------------------------------------------------------------------------

\section{Awards/Scholarships}

    \cventry{June 2016}
        {Best poster award - "Malignant diseases" category}{YSA Symposium}{}{}{Young Scientist Association of the Medical University of Vienna}
    \cventry{June 2016}
        {Best artwork award - "Illustrations and digital simulations" category}{Science\│Art Competition of the YSA Symposium}{}{}{Young Scientist Association of the Medical University of Vienna}
    \cventry{2013-2014}
        {Erasmus studies mobility program scholarship}{}{}{}{European Commission}
    \cventry{2011-2012}
        {Erasmus intership mobility program scholarship}{}{}{}{European Commission}
    \cventry{2009-2010}
        {"Integration into Research" Grant}{}{}{}{Science and Technology Foundation - Portugal}

%----------------------------------------------------------------------------------------
%   MEMBERSHIPS
%----------------------------------------------------------------------------------------

%\section{Associations}
%\cvitem{2015}{European Hematology Association (EHA)}

%----------------------------------------------------------------------------------------
%   LANGUAGES SECTION
%----------------------------------------------------------------------------------------

\section{Languages}

\cvitemwithcomment{Portuguese}{\textnormal{Native speaker}}{}
\cvitemwithcomment{English}{\textnormal{Very good}}{}
\cvitemwithcomment{Spanish}{\textnormal{Conversational}}{}
\cvitemwithcomment{German}{\textnormal{Basic}}{Basic words and phrases only}
\cvitemwithcomment{French}{\textnormal{Basic}}{Basic words and phrases only}

%----------------------------------------------------------------------------------------
%   INTERESTS SECTION
%----------------------------------------------------------------------------------------

\section{Other interests}

\cvlistdoubleitem{Ballroom dancing}{Cinema}
\cvlistdoubleitem{Singing}{Opera}
\cvlistdoubleitem{Choral conducting}{Piano}


%----------------------------------------------------------------------------------------
%   COVER LETTER
%----------------------------------------------------------------------------------------

% To remove the cover letter, comment out this entire block

% \clearpage

% \recipient{HR Departmnet}{Corporation\\123 Pleasant Lane\\12345 City, State} % Letter recipient
% \date{\today} % Letter date
% \opening{Dear Sir or Madam,} % Opening greeting
% \closing{Sincerely yours,} % Closing phrase
% \enclosure[Attached]{curriculum vit\ae{}} % List of enclosed documents

% \makelettertitle % Print letter title

% \lipsum[1-3] % Dummy text

% \makeletterclosing % Print letter signature

%----------------------------------------------------------------------------------------

\end{document}
